\documentclass[a4paper,12pt]{article}
\usepackage[slovene]{babel}
\usepackage[utf8]{inputenc}
\usepackage[T1]{fontenc}
\usepackage{lmodern}
\usepackage{amsmath, amsfonts, amsthm, amssymb}
\usepackage{a4wide}

\renewcommand\qedsymbol{$\blacksquare$}

\newtheorem{izrek}{Izrek}[section]

\theoremstyle{definition}
\newtheorem{definicija}{Definicija}[section]

\newtheorem{trditev}{Trditev}[section]

\theoremstyle{remark}
\newtheorem*{opomba}{Opomba}

\theoremstyle{definition}
\newtheorem*{primer}{Primer}


\title{%
  Metrična dimenzija usmerjenih grafov \\
  \large Projekt pri predmetu finančni praktikum\\}
     

\author{Aljaž Bratuša, Maša Popovič}
\date{10. 11. 2024}

\begin{document}
\maketitle

\section{Uvod in definicije}

Tema projektne naloge je metrična dimenzija usmerjenih grafov. V prvem delu želimo zapisati 
celoštevilski linearni program za določanje metrične dimenzije usmerjenih grafov. V drugem delu 
pa si podrobneje ogledamo posebno skupino grafov - cirkulantske grafe $C(n, d)$ s povezavami v smeri
urinega kazalca. Za majhne vrednosti parametrov $n$ in  $d$ želimo preveriti metrično dimenzijo, nato 
pa izpeljati splošen rezultat za grafe iz te skupine. Začnimo z nekaj osnovnimi pojmi
in definicijami. 

\begin{definicija}
    Naj bo $G$ usmerjen graf z množico vozlišč $V$ in 
    množico povezav $E$ ter $u,v \in V.$ Za vozlišče $v$ velja, da je \textit{dosegljivo}
    iz $u$, če med $u$ in $v$ obstaja pot. Za dolžino poti med vozliščema vzamemo dolžino
    najkrajše poti.
\end{definicija}

\begin{definicija}
    Naj bo $G$ graf z množico vozlišč $V$ in množico povezav $E$ ter $u,v,w \in V.$ Za vozlišče 
    $w$ pravimo, da \textit{razreši} par vozlišč $u$ in $v,$če velja, da sta $u$ in $v$ dosegljivi
    iz $w$ na različnih razdaljah, t.j. $d(u,w) \neq d(v,w).$ Množica vozlišč $S$ razreši graf $G,$ če 
    za vsak par vozlišč v $G$ velja, da ga razreši nek element $S.$ \textit{Metrična dimenzija} grafa $G$
    je kardinalnost najmanjše množice, ki zadošča prejšnji lastnosti. To označimo z $\beta(G).$
\end{definicija}

\begin{opomba}
    Množica $S,$ ki razreši dani graf $G$ ni nujno enolična. Preprost primer je npr. graf, ki je pot. Potem je 
    metrična dimenzija enaka $1,$ za vozlišče v $S$ pa lahko vzamemo eno izmed robnih vozlišč grafa $G.$
\end{opomba}

\begin{definicija}
    Naj bo $G$ graf z množico vozlišč $V$ in množico povezav $E.$
    Naj bo $W = \{w_1, w_2, \dots ,w_n \},  W \subseteq V$ urejena množica in $v \in V.$
    Oznaki $$r(v|W) = (d(v, w_1), d(v, w_2), \dots ,d(v, w_n))$$ pravimo metrična 
    reprezentacija $v$ glede na množico $W.$ Sledi, da $W$ razreši $G$ natanko tedaj, ko velja 
    $r(u|W) \neq r(v|W)$ za vse $u,v \in V,$ kjer velja $u \neq v.$
\end{definicija}

\begin{definicija}
    Za pozitivni celi števili $n$ in $d$ je cirkulantski graf $C(n, d)$ enostaven graf z vozlišči 
    ostankov po modulu $n,$ $Z_n = (v_0, v_1, \dots ,v_{n-1})$, kjer velja, da je $i-to$ vozlišče 
    sosedno vozliščem $v_{i-d}, v_{i-d+1}, \dots ,v_{i-1}, v_{i+1}, \dots ,v_{i-d+1}$ (mod $n$)
    v $C(n,d).$ \cite{Chau2017}
\end{definicija}

\section{Načrt za nadaljnje delo}

\begin{itemize}
    \item Zapisati celoštevilski linearni program za določanje metrične dimenzije usmerjenga grafa. 
    \item Poiskati metrične dimenzije naključnih usmerjenih grafov.
    \item Uporaba programa za določitev splošne formule za metrično dimenzijo $C(n, d).$
\end{itemize}

\bibliographystyle{plain}
\bibliography{viri}

\end{document}