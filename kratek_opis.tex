
\documentclass[a4paper,12pt]{article}
\usepackage[slovene]{babel}
\usepackage[utf8]{inputenc}
\usepackage[T1]{fontenc}
\usepackage{lmodern}
\usepackage{amsmath, amsfonts, amsthm, amssymb}
\usepackage{a4wide}

\bibliographystyle{plain}

\newtheorem{izrek}{Izrek}[section]



\theoremstyle{definition}
\newtheorem{definicija}{Definicija}[section]

\newtheorem{trditev}{Trditev}[section]

\theoremstyle{remark}
\newtheorem*{opomba}{Opomba}

\theoremstyle{definition}
\newtheorem*{primer}{Primer}


\title{
  Metrična dimenzija usmerjenih grafov \\
  \large Projekt pri predmetu finančni praktikum\\}
     

\author{Maša Popovič}
\date{10. 11. 2024}

\begin{document}
\maketitle

\section{Uvod in definicije}

V prvem delu želimo zapisati celoštevilski linearni program za določanje metrične dimenzije usmerjenih grafov. V drugem delu 
pa si podrobneje ogledamo posebno skupino grafov - cirkulantske grafe $C(n, d)$ s povezavami v smeri
urinega kazalca. Za majhne vrednosti parametrov $n$ in  $d$ želimo preveriti metrično dimenzijo, nato 
pa izpeljati splošen rezultat za grafe iz te skupine. Začnimo z nekaj osnovnimi pojmi
in definicijami. 

\begin{definicija}
    Naj bo $G$ usmerjen graf z množico vozlišč $V$ in 
    množico povezav $E$ ter $u,v \in V.$ Za vozlišče $v$ velja, da je \textit{dosegljivo}
    iz $u$, če med $u$ in $v$ obstaja pot. Za dolžino poti med vozliščema vzamemo dolžino
    najkrajše poti.
    \\
\end{definicija}


\begin{definicija}
    Naj bo $G$ graf z množico vozlišč $V$ in množico povezav $E$ ter $u,v,w \in V.$ Za vozlišče 
    $w$ pravimo, da \textit{razreši} par vozlišč $u$ in $v,$če velja, da sta $u$ in $v$ dosegljivi
    iz $w$ na različnih razdaljah, t.j. $d(w,u) \neq d(w,v).$ Množica vozlišč $S$ razreši graf $G,$ če 
    za vsak par vozlišč v $G$ velja, da ga razreši nek element $S.$ \textit{Metrična dimenzija} grafa $G$
    je kardinalnost najmanjše množice, ki zadošča prejšnji lastnosti. To označimo z $\beta(G).$
    \\
\end{definicija}

\begin{opomba}
    Množica $S,$ ki razreši dani graf $G$ ni nujno enolična. Preprost primer je npr. graf, ki je pot. Potem je 
    metrična dimenzija enaka $1,$ za vozlišče v $S$ pa lahko vzamemo eno izmed robnih vozlišč grafa $G.$
    \\
\end{opomba}

\begin{definicija}
    Naj bo $G$ graf z množico vozlišč $V$ in množico povezav $E.$
    Naj bo $W = \{w_1, w_2, \dots ,w_n \},  W \subseteq V$ urejena množica in $v \in V.$
    Oznaki $$r(v|W) = (d (w_1,v), d(w_2,v), \dots ,d(w_n,v))$$ pravimo metrična 
    reprezentacija $v$ glede na množico $W.$ Sledi, da $W$ razreši $G$ natanko tedaj, ko velja 
    $r(u|W) \neq r(v|W)$ za vse $u,v \in V,$ kjer velja $u \neq v.$
    \\
\end{definicija}


\begin{definicija}
    \textit{V smeri urinega kazalca usmerjen cirkulantni graf \( C(n, d) \) }je usmerjen graf z \( n \) vozlišči, kjer so vozlišča označena kot 
    \( v_0, v_1, \dots, v_{n-1} \). Množica povezav \( E \) vsebuje usmerjene povezave med vozlišči, in sicer vsakemu vozlišču \( v_i \) pripadajo povezave do naslednjih \( d \) vozlišč v smeri urinega kazalca. 
    Množica povezav je torej podana kot:
    \[
    E = \left\{ (v_i, v_{(i+k) \mod n}) : 1 \leq k \leq d \right\}, \quad \text{za vsak } i \in \{0, 1, \dots, n-1\}.
    \]
\end{definicija}

\section{Načrt za nadaljnje delo}

\begin{itemize}
    \item Zapisati celoštevilski linearni program za določanje metrične dimenzije usmerjenga grafa. 
    \item Poiskati metrične dimenzije $C(n, d).$ za majhne $n$ in $d$.
    \item Uporaba programa za določitev splošne formule za metrično dimenzijo $C(n, d).$
\end{itemize}



\subsection*{Trenutna ideja za CLP:}

\noindent
\textbf{Vhodni podatki:}
\begin{itemize}
    \item \textbf{Seznam vozlišč} \( V \): predstavimo kot seznam 
    \item \textbf{Matrika razdalj} \( D \): matrika, kjer je \( D[i][j] \) razdalja med vozliščema \( i \) in \( j \) v grafu. Razdalja je enaka  $\infty$ , če vozlišča niso neposredno povezana.
    Za izračun razdalj lahko na primer uporabimo algortem Floyd-Warshall ali BFS.
\end{itemize}

\subsubsection*{Ciljna funkcija:}
Minimiziramo število vozlišč v razrešljivi množici:
\begin{equation*}
\text{Minimiziraj:} \quad \sum_{v \in V} x_v
\end{equation*}
kjer je $x_v$ binarna spremenljivka, ki je enaka $1$, če je vozlišče $v$ vključeno v razrešljivo množico, in $0$, če ni.

\subsubsection*{Omejitve:}

\textbf{Pogoj razločljivosti:}
   Za vsak par različnih vozlišč $u$ in $v$ mora obstajati vsaj eno vozlišče $w$, ki jih razlikuje po razdalji.
   \begin{equation*}
   \sum_{w \in V} z_{u,v}^w \geq 1, \quad \forall u, v \in V, \; u \neq v
   \end{equation*}
   kjer je $z_{u,v}^w$ pomožna binarna spremenljivka, ki je enaka $1$, če vozlišče $w$ razlikuje par $(u, v)$ po razdalji.

\noindent
\textbf{Povezava med $z_{u,v}^w$ in $x_w$:}
   Pomožna spremenljivka $z_{u,v}^w$ lahko postane $1$ samo, če je vozlišče $w$ v razrešljivi množici (tj. če je $x_w = 1$). 
   \begin{equation*}
   z_{u,v}^w \leq x_w, \quad \forall u, v, w \in V, \; u \neq v
   \end{equation*}

\noindent
\textbf{Pogoji za binarne spremenljivke:}
    Vse spremenljivke morajo biti binarne, da zagotovimo celoštevilsko rešitev:
   \begin{equation*}
   x_v \in \{0, 1\}, \quad \forall v \in V
   \end{equation*}
   \begin{equation*}
   z_{u,v}^w \in \{0, 1\}, \quad \forall u, v, w \in V, \; u \neq v
   \end{equation*}


\end{document}